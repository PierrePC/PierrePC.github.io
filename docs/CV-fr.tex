\documentclass[11pt]{moderncv}

\moderncvtheme{classic}

\usepackage[utf8]{inputenc}
\usepackage[french]{babel}
\usepackage[scale=0.75]{geometry}
\usepackage{amssymb,amsmath}

\newcommand\Colorhref[2]{\href{#1}{\color{magenta}#2}}

\name{Pierre}{Perruchaud}
\title{Postdoctorant, Université de Notre Dame (États-Unis)}

\begin{document}
\makecvtitle
\let\thefootnote\relax\footnotetext{Dernière mise à jour le \today.}

\vspace{-25pt}

\section{Information}
%%%%%%%%%%%%%%%%%%%%%
\cvitem{État civil}{Né le 29 juin 1993 à Remiremont (88).}
\cvitem{Adresse mail}{\Colorhref{mailto:Pierre Perruchaud <pperruch@nd.edu>}{pperruch@nd.edu}}
\cvitem{Page web}{\Colorhref{https://pierrepc.github.io/index.html}{pierrepc.github.io}}
\cvitem{Langues}{Français: langue maternelle --- Anglais ($\simeq\text{C1}$) --- Allemand ($\simeq\text{B1}$).}

\section{Formation et emploi}
%%%%%%%%%%%%%%%%%%%

\cventry{2019 --- \phantom{2020}}{Visiting assistant professor}{University of Notre Dame}{États-Unis}{}
{Contrat postdoctoral.}
\cventry{2016 --- 2019}{Doctorat de mathématiques}{IRMAR, Université de Rennes 1}{}{}
{``Homogenisation for Kinetic Brownian Motion,''
\newline sous la direction de \href{https://perso.univ-rennes1.fr/jurgen.angst/}{Jürgen Angst} et \href{https://perso.univ-rennes1.fr/ismael.bailleul/}{Ismaël Bailleul}.
\newline Soutenance le 21 octobre 2019.}
\cventry{2012 --- 2016}{Élève à l'ENS de Lyon}{}{}{}{}
\cvitem{}{Durant ces quatre ans, j'ai effectué trois stages de recherche sous la direction de \href{http://www.lmpt.univ-tours.fr/~gouere/}{J.-B.~Gouéré} (6 semaines), \href{https://perso.univ-rennes1.fr/jurgen.angst/}{J.~Angst} (2 mois) et \href{http://www.math.sciences.univ-nantes.fr/~gouezel/}{S.~Gouëzel} (4 mois).}
   \cvitem{\textcolor{color1}{2016}}{Master 2.\newline Processus stochastiques et géométrie symplectique.}
   \cvitem{\textcolor{color1}{2015}}{Préparation à l'agrégation. Reçu 19\ieme.}
   \cvitem{\textcolor{color1}{2014}}{Master 1}
   \cvitem{\textcolor{color1}{2013}}
{Licence 3.}
\cventry{2010 --- 2012}{CPGE}{Bordeaux}{France}{}{}

\section{Recherche}
%%%%%%%%%%%%%%%%%%

\textbf{Mots clefs.} Analyse stochastique, chemins rugueux, géométrie riemannienne, analyse sur les variétés, noyaux de la chaleur, méthode de la paramétrix.

\medskip

Je m'intéresse aux interactions entre probabilités et géométrie, et plus précisément aux processus stochastiques à valeurs dans des variétés. Une partie de mon travail concerne les espaces de difféomorphismes, vus comme des groupes de Lie de dimension infinie. Dans cet esprit, mon résultat principal en date est la construction d'une perturbation aléatoire des équations d'Euler de la mécanique des fluides, et la preuve de sa convergence vers un mouvement brownien lorsque le paramètre de bruit diverge.

\smallskip

Dans une veine plus analytique, j'ai étudié le noyau associé à certains mouvements dégénérés. Le générateur infinitésimal d'une diffusion hypoelliptique contient une information géométrique complexe, que l'on peut (essayer de) relier aux propriétés analytiques du noyau associé. Un exemple classique dans le cas elliptique est le lien entre courbure et asymptotique en temps petit du noyau de la chaleur usuel.

\subsection{Articles}
\cventry{2019}
   {Kinetic Brownian motion on the diffeomorphism group of a closed Riemannian manifold}{}{}{}
   {Avec J.~Angst et I.~Bailleul. \texttt{arXiv:1905.04103}, 2019. \\ \Colorhref{https://arxiv.org/abs/1905.04103}{Consulter en ligne.}}
\cventry{2018}
   {Homogenisation for anisotropic kinetic random motion}{}{}{}
   {Electron. J. Probab., Volume 25 (2020), article no. 39, 26 pp. \\ \Colorhref{https://projecteuclid.org/euclid.ejp/1585620094}{Consulter en ligne.}}

\subsection{Séjour de recherche}

\cventry{Jan-Jun 2018}{University of Warwick}{}{Royaume-Uni}{}{Collaboration avec \href{https://warwick.ac.uk/fac/sci/statistics/staff/academic-research/kolokoltsov/}{Vassili Kolokotsov}.}

\subsection{Exposés lors de séminaires}

\cventry{Nov 2019}{Séminaire d'analyse et probabilités}{University of Connecticut}{États-Unis}{}{}
\cventry{Sep 2019}{Séminaire de géométrie}{University of Notre Dame}{États-Unis}{}{}
\cventry{Mai 2019}{Séminaire de probabilités}{IMT, Université de Toulouse III}{}{}{}
\cventry{Mar 2019}{Séminaire d'analyse}{IMB, Université de Bordeaux}{}{}{}
\cventry{Jan 2019}{Séminaire de probabilités}{IECL, Université de Lorraine}{}{}{}
\cventry{Fev 2019}{Séminaire de probabilités}{IRMAR, Université de Rennes 1}{}{}{}

\subsection{Exposés lors de conférences}

\cventry{Mai 2020}{\textsuperscript{*}~Recent advances in stochastic analysis}{Università di Pisa}{Italie}{}{}
\cventry{Jul 2020}{\textsuperscript{*}~Stochastic differential geometry and Mathematical physics}{Université de Rennes 1}{}{}{}

\smallskip

{\small\textsuperscript{*}~Reporté suite au Covid-19.}

\subsection{Participations à des conférences}
\cventry{Mar 2019}
   {New Directions in Stochastic Analysis}{Weierstrass Institute}{Allemagne}{}
   {Exposé de I.~Bailleul (directeur  de thèse) sur des travaux communs.}
\cventry{Jul 2017}
   {GeoProb 2017}{Université du Luxembourg}{Luxembourg}{}{}
\cventry{Jun 2017}
   {Journées de probabilités}{Centre Langevin}{}{}{}
\cventry{Jun 2016}
   {Singular Phenomena and Singular Geometries}
   {Università di Pisa}{Italie}{}{}

\section{Enseignement}
%%%%%%%%%%%%%%%%%%

J'ai été reçu 19\ieme\ au concours de l'agrégation en 2015. Depuis 2019, je suis en disponibilité, rattaché à l'académie de Dijon.

\medskip

% 2019-2020
\cventry{2019 --- 2020}
   {Introduction aux probabilités}
   {Univ. Notre Dame}{États-Unis}{}{Cours pour divers sections, en particulier mathématiques ; 2\ieme\ à 4\ieme\ année.}
\cventry{2019 --- 2020}
   {Analyse élémentaire}
   {Univ. Notre Dame}{États-Unis}{}{Cours pour divers sections, 1\iere\ année.}
% 2018-2019
\cventry{2018 --- 2019}
   {TP d'outils logiciels}
   {IUT Rennes 1}{}{}{Calculs scientifique et symbolique élémentaires. L1 et L2 d'informatique et électronique.}
\cventry{             }
   {TP de probabilités et statistiques}
   {Univ. Rennes 1}{}{}{L2 de mathématiques.}
% 2017-2018
\cventry{Aut. 2017}
   {TD de calcul vectoriel}
   {Université de Rennes 1}{}{}{L2 de physique.}
\cventry{         }
   {TP d'outils logiciels}
   {IUT Rennes 1}{}{}{}
% 2016-2017
\cventry{Mar 2017}
   {Systèmes dynamiques topologiques}
   {Universiteti i Prishtinës}{Kosovo}{}
   {Cours intensif de 8h à Pristina, dans le cadre d'une école de printemps pour étudiants kosovares sélectionnés organisée par l'ambassade française et l'Université de Lyon.}
\cventry{Prin. 2017}
   {TD d'Algèbre linéaire}
   {Université de Rennes 1}{}{}{L1 d'économie et mathématiques.}
\cventry{2016 --- 2017}
   {TP d'outils logiciels}
   {IUT Rennes 1}{}{}{}

\section{Autres activités scientifiques}

\cventry{Dec 2018}{Exposé aux `5 minutes Lebesgue'}{Centre Henri Lebesgue}{Rennes}{}
   {Un (très) rapide aperçu de la théorie des chemins rugueux.\newline Disponible sur \Colorhref{https://youtu.be/r31qisjgbK8}{YouTube}.}
\cventry{2018 --- 2019}
   {Organisateur du séminaire des doctorants en probabilités}
   {IRMAR}{Rennes}{}
   {}
\cventry{Fall 2017}
   {Co-organisateur du séminaire des doctorants en probabilités}
   {IRMAR}{Rennes}{}
   {Avec \href{http://perso.eleves.ens-rennes.fr/people/florian.lemonnier/}{Florian Lemonnier}.}

\end{document}
