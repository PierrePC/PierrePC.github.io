\documentclass[11pt,classic]{moderncv}

\moderncvtheme{classic}

\usepackage[utf8]{inputenc}
\usepackage[english]{babel}
\usepackage[scale=0.75]{geometry}
\usepackage{footmisc}
\usepackage{amssymb}
\usepackage[normalem]{ulem}

\AtBeginDocument{\hypersetup{colorlinks=true}}
\let\thefootnote\relax

\newcommand{\cvsubentry}[2]{\cvitem{\textcolor{color1}{#1}}{#2}}

\name{Pierre}{Perruchaud}
\title{Visiting assistant professor, University of Notre Dame}

\begin{document}
\makecvtitle
\footnotetext{Last updated \today.}

\vspace{-25pt}

\section{Information}
%%%%%%%%%%%%%%%%%%%%%
\cvitem{Civil status}{Born June 29, 1993 in Remiremont, France.}
\cvitem{Email}{\href{mailto:Pierre Perruchaud <pperruch@nd.edu>}{pperruch@nd.edu}}
\cvitem{Website}{\href{https://pierrepc.github.io/index.html}{pierrepc.github.io}}
\cvitem{Languages}{French: native speaker --- English: fluent --- German: working knowledge.}

\section{Employment and Education}
%%%%%%%%%%%%%%%%%%%

\cventry{2019 --- \phantom{2020}}{Visiting assistant professor}{University of Notre Dame}{USA}{}
        {Temporary postdoctoral position; includes a two-course teaching load per semester.}
\cventry{2016 --- 2019}{PhD student in mathematics}{IRMAR, Université de Rennes 1}{France}{}
        {``Homogenisation for Kinetic Brownian Motion''\\Under the supervision of Jürgen Angst and Ismaël Bailleul. Defended in October 2019.}
\cventry{2012 --- 2016}{Student of the ENS de Lyon}{}{France}{}
   {Including three research internships, under the supervision of \href{http://www.lmpt.univ-tours.fr/~gouere/}{J.-B.~Gouéré} (6 weeks), \href{https://perso.univ-rennes1.fr/jurgen.angst/}{J.~Angst} (2 months) and \href{http://www.math.sciences.univ-nantes.fr/~gouezel/}{S.~Gouëzel} (4 months). \newline
    Successful candidate of the French high school teacher's competitive exam (\href{https://en.wikipedia.org/wiki/Agr\%C3\%A9gation}{agrégation}), ranking 19th.}

\section{Research}

\textbf{Keywords.} Stochastic analysis, rough paths theory, Riemannian geometry, global analysis, WKB method.

\smallskip

I am interested in the interaction between probability and geometry, specifically manifold-valued stochastic processes. Part of my work deals with diffusions with values in spaces of diffeomorphisms, seen as Lie groups of infinite dimension. In this direction, my most noteworthy result to date is the construction and study of a random perturbation of the Euler equations of fluid mechanics.

\smallskip

In a more analytical mindset, I have studied the kernel associated to some degenerate motions. The infinitesimal generator of a hypoelliptic diffusion contains subtle geometric data, which we can try to relate to the analytical properties of the kernel, somewhat similarly to the link between curvature and small time asymptotics of the usual heat kernel.

\subsection{Publications}
\cventry{2020}
   {Homogenisation for anisotropic kinetic random motion}{}{}{}
   {Electron. J. Probab. 25 (2020), paper no. 39. \href{https://projecteuclid.org/euclid.ejp/1585620094}{Available online}.}
\subsection{Preprints}
\cventry{2021}
   {Kinetic Dyson Brownian motion}{}{}{}
   {Available at \href{https://arxiv.org/abs/2101.10426}{arXiv:2101.10426}.}
\cventry{2019}
   {Kinetic Brownian motion on the diffeomorphism group of a closed Riemannian manifold}{with J.~Angst and I.~Bailleul}{}{}
   {Available at \href{https://arxiv.org/abs/1905.04103}{arXiv:1905.04103}.}
\subsection{PhD Dissertation}
\cventry{2019}{Homogenisation for Kinetic Brownian Motion}{}{under the supervision of \href{https://perso.univ-rennes1.fr/jurgen.angst/}{Jürgen Angst} and \href{https://perso.univ-rennes1.fr/ismael.bailleul/}{Ismaël Bailleul}}{}{Available on \href{https://pierrepc.github.io/docs/These.pdf}{my website} and on the official French \href{http://www.theses.fr/2019REN1S057}{theses.fr} website.}

\subsection{Research stay}

\cventry{Jan-Jun 2018}{University of Warwick}{}{United Kingdom}{}{Collaboration with \href{https://warwick.ac.uk/fac/sci/statistics/staff/academic-research/kolokoltsov/}{Vassili Kolokotsov}.}

\subsection{Talks in seminars}

\cventry{Nov 2019}{Analysis and probability seminar}{University of Connecticut}{USA}{}{}
\cventry{Sep 2019}{Probability seminar}{University of Notre Dame}{USA}{}{}
\cventry{May 2019}{Probability seminar}{IMT, Université de Toulouse III}{France}{}{}
\cventry{Mar 2019}{Analysis seminar}{IMB, Université de Bordeaux}{France}{}{}
\cventry{Jan 2019}{Probability seminar}{IECL, Université de Lorraine}{France}{}{}
\cventry{Feb 2019}{Probability seminar}{IRMAR, Université de Rennes 1}{France}{}{}

\subsection{Talks in conferences}

\cventry{Jun 2020}{Stochastic Analysis Brats}{Università di Pisa}{Italy}{}{}
\cventry{Jul 2020}{\sout{Stochastic differential geometry and Mathematical physics}}{Université de Rennes 1}{France}{}{Postponed due to coronavirus.}

\subsection{Conferences attended}
\cventry{Mar 2019}
   {New Directions in Stochastic Analysis}{Weierstrass Institute}{Germany}{}
   {Talk by my PhD advisor about joint work.}
\cventry{Jul 2017}
   {GeoProb 2017}{Université du Luxembourg}{Luxembourg}{}{}
\cventry{Jun 2017}
   {Journées de probabilités}
   {Centre Langevin}{France}{}{Conference for young researchers in probability (literally: Days of probability).}
\cventry{Jun 2016}
   {Singular Phenomena and Singular Geometries}
   {Università di Pisa}{Italy}{}{}

\section{Teaching}
%%%%%%%%%%%%%%%%%%

I hold a higher teaching diploma, the French \href{https://en.wikipedia.org/wiki/Agr\%C3\%A9gation}{agrégation}.

\medskip

% Notre Dame
\cventry{2019 --- 2021}{University of Notre Dame}{USA}{Instructor}{}{}
\cvsubentry{Sping 2021}{Calculus for mixed undergraduates.\newline%
                        Elementary probability for mathematics undergraduates.}
\cvsubentry{Fall 2020}{Calculus for mixed undergraduates, two sections.}
\cvsubentry{Spring 2020}{Elementary probability for mathematics undergraduates.}
\cvsubentry{Fall 2019}{Calculus for mixed undergraduates, two sections.}
% Rennes
\cventry{2016 --- 2019}{Université de Rennes 1}{France}{Teaching Assistant}{}{}
\cvsubentry{Spring 2019}{Computer practicals for engineering undergraduates, two sections.\newline%
                         Computer practicals for mathematics undergraduates.}
\cvsubentry{Fall 2018}{Computer practicals for engineering undergraduates, two sections.}
\cvsubentry{Fall 2017}{Computer practicals for engineering undergraduates, three sections.\newline%
                       Vector calculus for physics undergraduates}
\cvsubentry{Spring 2017}{Computer practicals for engineering undergraduates.\newline%
                         Linear algebra for economics undergraduates.}
\cvsubentry{Fall 2016}{Computer practicals for engineering undergraduates.}
% Prishtinë
\cventry{Mar 2017}{Universiteti i Prishtinës}{Kosovo}{Instructor}{}
   {Topological dynamical systems for selected undergraduates in mathematics. \newline
    Intensive 8-hour lectures in English, part of a spring school.}

\section{Other scientific activities}

\cventry{Dec 2018}{`5 minutes Lebesgue' talk}{Centre Henri Lebesgue}{Rennes}{}
   {A glimpse of rough paths theory, part of a series of (very) short talks.\newline Available on \href{https://youtu.be/r31qisjgbK8}{YouTube} (in French).}
\cventry{2018 --- 2019}
   {Organiser of the PhD seminar in probability}
   {IRMAR}{Rennes}{}
   {}
\cventry{Fall 2017}
   {Co-organiser of the PhD seminar in probability}
   {IRMAR}{Rennes}{}
   {In collaboration with \href{http://perso.eleves.ens-rennes.fr/people/florian.lemonnier/}{Florian Lemonnier}.}

\end{document}
