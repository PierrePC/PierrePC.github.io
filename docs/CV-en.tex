\documentclass[11pt,classic]{moderncv}

\moderncvtheme{classic}

\usepackage[utf8]{inputenc}
\usepackage[english]{babel}
\usepackage[scale=0.75]{geometry}
\usepackage{amssymb}

\name{Pierre}{Perruchaud}
\title{Visiting assistant professor, University of Notre Dame}

\begin{document}
\makecvtitle
\footnotetext{Last updated \today.}

\vspace{-25pt}

\section{Information}
%%%%%%%%%%%%%%%%%%%%%
\cvitem{Civil status}{Born June 29, 1993 in Remiremont, France.}
\cvitem{Email}{\href{mailto:Pierre Perruchaud <pperruch@nd.edu>}{pperruch@nd.edu}}
\cvitem{Website}{\href{https://pierrepc.github.io/index.html}{pierrepc.github.io}}
\cvitem{Languages}{French: native speaker --- English: fluent --- German: working knowledge.}

\section{Employment and Education}
%%%%%%%%%%%%%%%%%%%

\cventry{2019 --- \phantom{2020}}{Visiting assistant professor}{University of Notre Dame}{USA}{}
{Temporary postdoctoral position.}
\cventry{2016 --- 2019}{PhD student in mathematics}{IRMAR, Université de Rennes 1}{France}{}
{``Homogenisation for Kinetic Brownian Motion,''
\newline under the supervision of \href{https://perso.univ-rennes1.fr/jurgen.angst/}{Jürgen Angst} and \href{https://perso.univ-rennes1.fr/ismael.bailleul/}{Ismaël Bailleul}.
\newline Defended on the 21st of October, 2019.}
\cventry{2012 --- 2016}{Student of the \href{https://en.wikipedia.org/wiki/\%C3\%89cole_normale_sup\%C3\%A9rieure}{ENS de Lyon}}{}{France}{}{}
\cvitem{}{During these four years I did three research internships, under the supervision of \href{http://www.lmpt.univ-tours.fr/~gouere/}{J.-B.~Gouéré} (6 weeks), \href{https://perso.univ-rennes1.fr/jurgen.angst/}{J.~Angst} (2 months) and \href{http://www.math.sciences.univ-nantes.fr/~gouezel/}{S.~Gouëzel} (4 months).}
   \cvitem{\textcolor{color1}{2016}}
{Second year of master's degree (French M2).\newline
 Symplectic geometry and stochastic processes.}
   \cvitem{\textcolor{color1}{2015}}
{Preparation for the French high school teacher's competitive exam (French \href{https://en.wikipedia.org/wiki/Agr\%C3\%A9gation}{agrégation}). Successful candidate, ranked 19.}
   \cvitem{\textcolor{color1}{2014}}
{First year of master's degree (French M1).}
   \cvitem{\textcolor{color1}{2013}}
{Bachelor's degree (French L3).}
\cventry{2010 --- 2012}{Preparatory classes (French \href{https://en.wikipedia.org/wiki/Classe_pr\%C3\%A9paratoire_aux_grandes_\%C3\%A9coles}{CPGE})}{Bordeaux}{France}{}
{Two-year intensive preparation to the exams of the French \href{https://en.wikipedia.org/wiki/Grandes_\%C3\%A9coles}{\emph{grandes écoles.}}\newline
 Mathematics with a minor in physics.}

\smallskip

{\small(Some items above are quite French-specific; I added Wikipedia links, should anything be unclear.)}

\section{Research}

\textbf{Keywords.} Stochastic analysis, rough paths theory, Riemannian geometry, global analysis, WKB method.

\medskip

I am interested in the interaction between probability and geometry, specifically manifold-valued stochastic processes. Part of my work deals with diffusions with values in spaces of diffeomorphisms, seen as Lie groups of infinite dimension. In this direction, my most noteworthy result to date is the construction of a random perturbation of the Euler equations of fluid mechanics, and the proof of its convergence to Brownian motion as some noise parameter diverges.

\smallskip

In a more analytical mindset, I have studied in my research the kernel associated to some degenerate motions. The infinitesimal generator of a hypoelliptic diffusion contains subtle geometric data, which we can (try to) relate to the analytical properties of the kernel; a well-known (ellitic) example is the link between curvature and small time asymptotics of the usual heat kernel.

\subsection{Articles}
\cventry{2019}
   {Kinetic Brownian motion on the diffeomorphism group of a closed Riemannian manifold}{J.~Angst, I.~Bailleul, P.~Perruchaud}{}{}
   {Available at \href{https://arxiv.org/abs/1905.04103}{arXiv:1905.04103}.}
\cventry{2018}
   {Homogenisation for anisotropic kinetic random motion}{P.~Perruchaud}{}{}
   {To appear in Electronic Journal of Probability.\\Draft available at \href{https://arxiv.org/abs/1811.08415}{arXiv:1811.08415}.}

\subsection{Research stay}

\cventry{Jan-Jun 2018}{University of Warwick}{}{United Kingdom}{}{Collaboration with \href{https://warwick.ac.uk/fac/sci/statistics/staff/academic-research/kolokoltsov/}{Vassili Kolokotsov}.}

\subsection{Talks in seminars}

\cventry{Nov 2019}{Analysis and probability seminar}{University of Connecticut}{USA}{}{}
\cventry{Sep 2019}{Probability seminar}{University of Notre Dame}{USA}{}{}
\cventry{May 2019}{Probability seminar}{IMT, Université de Toulouse III}{France}{}{}
\cventry{Mar 2019}{Analysis seminar}{IMB, Université de Bordeaux}{France}{}{}
\cventry{Jan 2019}{Probability seminar}{IECL, Université de Lorraine}{France}{}{}
\cventry{Feb 2019}{Probability seminar}{IRMAR, Université de Rennes 1}{France}{}{}

\subsection{Talks in conferences}

\cventry{May 2020}{\textsuperscript{*}~Recent advances in stochastic analysis}{Università di Pisa}{Italy}{}{}
\cventry{Jul 2020}{\textsuperscript{*}~Stochastic differential geometry and Mathematical physics}{Université de Rennes 1}{France}{}{}

\smallskip

{\small\textsuperscript{*}~Postponed due to coronavirus.}

\subsection{Conferences attended}
\cventry{Mar 2019}
   {New Directions in Stochastic Analysis}{Weierstrass Institute}{Germany}{}
   {Talk by my PhD advisor about joint work.}
\cventry{Jul 2017}
   {GeoProb 2017}{Université du Luxembourg}{Luxembourg}{}{}
\cventry{Jun 2017}
   {Journées de probabilités}
   {Centre Langevin}{France}{}{Conference for young researchers in probability (literally: Days of probability).}
\cventry{Jun 2016}
   {Singular Phenomena and Singular Geometries}
   {Università di Pisa}{Italy}{}{}

\section{Teaching}
%%%%%%%%%%%%%%%%%%

I hold a higher teaching diploma, the French \href{https://en.wikipedia.org/wiki/Agr\%C3\%A9gation}{\emph{agrégation;}} see the Education section above.

\medskip

% 2019-2020
\cventry{2019 --- 2020}
   {Probability for maths undergraduates}
   {University of Notre Dame}{}{}{Elective course.}
\cventry{2019 --- 2020}
   {Calculus for mixed undergraduates}
   {University of Notre Dame}{}{}{}
% 2018-2019
\cventry{2018 --- 2019}
   {Computer practicals for engineering undergraduates}
   {IUT de Rennes 1}{}{}{}
% 2017-2018
\cventry{Fall 2017}
   {Vector calculus, physics undergraduates}
   {Université de Rennes 1}{}{}{}
\cventry{         }
   {Computer practicals for engineering undergraduates}
   {IUT de Rennes 1}{}{}{}
% 2016-2017
\cventry{Mar 2017}
   {Topological dynamical systems course}
   {Universiteti i Prishtinës}{Kosovo}{}
   {Intensive 8-hour lectures in Pristina, part of a spring school for selected Kosovar undergraduates. In collaboration with the French embassy and the Université de Lyon.}
\cventry{Spring 2017}
   {Linear algebra, economics undergraduates}
   {Université de Rennes 1}{}{}{}
\cventry{2016 --- 2017}
   {Computer practicals for engineering undergraduates}
   {IUT de Rennes 1}{}{}{}

\section{Other scientific activities}

\cventry{Dec 2018}{`5 minutes Lebesgue' talk}{Centre Henri Lebesgue}{Rennes}{}
   {A glimpse of rough paths theory, part of a series of (very) short talks.\newline Available on \href{https://youtu.be/r31qisjgbK8}{YouTube} (fr).}
\cventry{2018 --- 2019}
   {Organiser of the PhD seminar in probability}
   {IRMAR}{Rennes}{}
   {}
\cventry{Fall 2017}
   {Co-organiser of the PhD seminar in probability}
   {IRMAR}{Rennes}{}
   {In collaboration with \href{http://perso.eleves.ens-rennes.fr/people/florian.lemonnier/}{Florian Lemonnier}.}

\end{document}
