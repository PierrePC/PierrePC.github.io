\documentclass[11pt,classic,colorlinks]{moderncv}

\moderncvtheme{classic}

\usepackage[utf8]{inputenc}
\usepackage[english]{babel}
\usepackage[scale=0.75]{geometry}
\usepackage{footmisc}
\usepackage{amssymb}
\usepackage[normalem]{ulem}

%\AtBeginDocument{\hypersetup{colorlinks=true}}
\let\thefootnote\relax

\newcommand{\cvsubentry}[2]{\cvitem{\textcolor{color1}{#1}}{#2}}

\name{Pierre}{Perruchaud}
\title{Postdoctoral researcher, Université du Luxembourg}

\begin{document}
\makecvtitle
\footnotetext{Last updated \today.}

\vspace{-25pt}

\section{Information}
%%%%%%%%%%%%%%%%%%%%%
\cvitem{Civil status}{French nationality. Born June 29, 1993 in Remiremont, France.}
\cvitem{Email}{\href{mailto:Pierre Perruchaud <pierre.perruchaud@uni.lu>}{pierre.perruchaud@uni.lu}}
\cvitem{Website}{\href{https://pierrepc.github.io/index.html}{pierrepc.github.io}}

\section{Employment and Education}
%%%%%%%%%%%%%%%%%%%

\cventry{2021 --- \phantom{2022}}{Postdoctoral researcher}{Université du Luxembourg}{}{}
        {Temporary postdoctoral position; includes a teaching load.}
\cventry{2019 --- 2021}{Visiting assistant professor}{University of Notre Dame}{USA}{}
        {Temporary postdoctoral position; includes a two-course teaching load per semester.}
\cventry{2016 --- 2019}{PhD student in mathematics}{IRMAR, Université de Rennes 1}{France}{}
        {``Homogenisation for Kinetic Brownian Motion''\\Under the supervision of Jürgen Angst and Ismaël Bailleul. Defended in October 2019.}
\cventry{2012 --- 2016}{Student of the ENS de Lyon}{}{France}{}
   {Including three research internships, under the supervision of \href{http://www.lmpt.univ-tours.fr/~gouere/}{J.-B.~Gouéré} (6 weeks), \href{https://perso.univ-rennes1.fr/jurgen.angst/}{J.~Angst} (2 months) and \href{http://www.math.sciences.univ-nantes.fr/~gouezel/}{S.~Gouëzel} (4 months). \newline
    Successful candidate of the French high school teacher's competitive exam (\href{https://en.wikipedia.org/wiki/Agr\%C3\%A9gation}{agrégation}), ranking 19th.}

\section{Research}

\textbf{Keywords.} Stochastic processes on manifolds, Malliavin calculus, rough paths theory, differential topology, infinite-dimensional topology, global analysis.

\medskip

I am interested in stochastic processes with values in manifolds and function spaces. In my work, those target spaces have been Lie groups, in which case we are talking about random matrices; diffeomorphisms of a base space, which corresponds to random fluids; sections or connections of a vector bundle, representing fermionic and bosonic fields. In their study, I am lead to use tools from stochastic calculus and analysis (contracting semigroups, partial differential equations, rough path theory, Malliavin calculus...) with methods of differential geometry (principal bundles, index theory, differential topology, singularity theory...).
\smallskip

This perspective applies to many problems, but I am particularly inspired by two types of questions. First, the (rigorous) construction and study of models coming from physics, for instance in gauge theory or in fluid mechanics. Second, the exploration of the dialogue between the geometry of a space and the processes that evolve on them; for me, those are rooted in the idea that Brownian motion tends to be pushed away from points with negative curvature.
\newpage

\subsection{Publications}
\cventry{2025?}
   {Kinetic Brownian motion on the diffeomorphism group of a closed Riemannian manifold}{with J.~Angst and I.~Bailleul}{}{}
   {To appear in Ann. Sc. Norm. Super. Pisa Cl. Sci. (5).\\
    arXiv version: \href{https://arxiv.org/abs/1905.04103}{arxiv:1905.04103}}
\cventry{2022}
   {Kinetic Dyson Brownian motion}{}{}{}
   {Electron. Commun. Probab. 27 (2022), article no. 37. \href{ https://doi.org/10.1214/22-ECP480}{Freely available online}.}
\cventry{2020}
   {Homogenisation for anisotropic kinetic random motion}{}{}{}
   {Electron. J. Probab. 25 (2020), article no. 39. \href{ https://doi.org/10.1214/20-EJP439}{Freely available online}.}
\subsection{Preprints}
\cventry{2024}
   {Loop soup representation of zeta-regularised determinants and equivariant Symanzik identities}{with I.~Sauzedde}{}{}
   {Available at \href{https://arxiv.org/abs/2402.00767}{2402.00767}.}
   \cventry{2023}
   {Small time expansion for a strictly hypoelliptic kernel}{}{}{}
   {Available at \href{https://arxiv.org/abs/2301.06904}{2301.06904}.}
\subsection{PhD Dissertation}
\cventry{2019}{Homogenisation for Kinetic Brownian Motion}{}{under the supervision of \href{https://perso.univ-rennes1.fr/jurgen.angst/}{Jürgen Angst} and \href{https://perso.univ-rennes1.fr/ismael.bailleul/}{Ismaël Bailleul}}{}{Available on \href{https://pierrepc.github.io/docs/These.pdf}{my website} and on the official French \href{http://www.theses.fr/2019REN1S057}{theses.fr} website.}

\subsection{Research stay}

\cventry{Jan-Jun 2018}{University of Warwick}{}{United Kingdom}{}{Collaboration with \href{https://warwick.ac.uk/fac/sci/statistics/staff/academic-research/kolokoltsov/}{Vassili Kolokotsov}.}

\subsection{Talks in seminars}

\cventry{Nov 2024*}{Harmonic analysis seminar}{LMO, Université Paris-Saclay}{France}{}{}
\cventry{Nov 2024*}{Probability seminar}{Université Grenoble Alpes}{France}{}{}
\cventry{Feb 2024}{Riemannian geometry and stochastics workgroup}{TU Delft}{Netherlands}{}{}
\cventry{Dec 2023}{Probability seminar}{MAP5, Université Paris Cité}{France}{}{}
\cventry{May 2022}{Work in Progress seminar (probability and statistics)}{}{Luxembourg}{}{}
\cventry{Nov 2021}{Work in Progress seminar (probability and statistics)}{}{Luxembourg}{}{}
\cventry{Nov 2019}{Analysis and probability seminar}{University of Connecticut}{USA}{}{}
\cventry{Sep 2019}{Probability seminar}{University of Notre Dame}{USA}{}{}
\cventry{May 2019}{Probability seminar}{IMT, Université de Toulouse III}{France}{}{}
\cventry{Mar 2019}{Analysis seminar}{IMB, Université de Bordeaux}{France}{}{}
\cventry{Jan 2019}{Probability seminar}{IECL, Université de Lorraine}{France}{}{}
\cventry{Feb 2019}{Probability seminar}{IRMAR, Université de Rennes 1}{France}{}{}

\subsection{Talks in conferences and workshops}

\cventry{Sept 2024}{Joint DPhysMS and DMath workshop}{Université du Luxembourg}{Luxembourg}{}{}
\cventry{Sept 2024}{L\textsuperscript2 Workshop in Probability and Statistics}{Université de Lorraine and Université du Luxembourg}{France}{}{}
\cventry{Sept 2024}{Interacting particles in the continuum}{EURANDOM, TU Eindhoven}{Netherlands}{}{}
\cventry{Feb 2023}{Luxembourg Workshop in Stochastic Analysis}{Université du Luxembourg}{}{}{}
\cventry{Feb 2023}{Luxembourg Workshop in Stochastic Analysis}{Université du Luxembourg}{}{}{}
\cventry{Sep 2021}{SchröMoka conference}{Universidade de Lisoa}{Portugal}{}{}
\cventry{Jul 2021}{Stochastic differential geometry and Mathematical physics}{Université de Rennes 1}{France}{}{}
\cventry{Mar 2021}{Stochastics and Geometry}{Banff International Research Station}{Canada}{}{}
\cventry{Jun 2020}{Stochastic Analysis Brats}{Università di Pisa}{Italy}{}{}

\section{Teaching}
%%%%%%%%%%%%%%%%%%

I hold a higher teaching diploma, the French \href{https://en.wikipedia.org/wiki/Agr\%C3\%A9gation}{agrégation}.

\subsection{Lectures}

% Luxembourg
\cventry{2021 -- 2024}{Université du Luxembourg}{}{}{}{}
\cvsubentry{Fall 2024}{Elementary topology, bachelor.\newline%
                       Measure theory crash course, master.}
\cvsubentry{Spring 2024}{Markov chains, bachelor.}
\cvsubentry{Fall 2023}{Complex analysis crash course, bachelor.}
\cvsubentry{Spring 2023}{Markov chains, bachelor.}
\cvsubentry{Fall 2022}{Elementary probability, bachelor (physics).}
\cvsubentry{Spring 2022}{Markov chains, bachelor.}
\cvsubentry{Fall 2021}{Elementary probability, bachelor (physics).}
% Notre Dame
\cventry{2019 --- 2021}{University of Notre Dame}{USA}{}{}{}
\cvsubentry{Spring 2021}{Calculus for mixed undergraduates.\newline%
                         Elementary probability for mathematics undergraduates.}
\cvsubentry{Fall 2020}{Calculus for mixed undergraduates, two sections.}
\cvsubentry{Spring 2020}{Elementary probability for mathematics undergraduates.}
\cvsubentry{Fall 2019}{Calculus for mixed undergraduates, two sections.}
% Prishtinë
\cventry{Mar 2017}{Universiteti i Prishtinës}{Kosovo}{}{}
   {Topological dynamical systems for selected undergraduates in mathematics. \newline
    Intensive 8-hour lectures in English, part of a spring school.}

\subsection{Exercise sessions and practicals}

% Luxembourg
\cventry{2021 -- 2022}{Université du Luxembourg}{}{}{}{}
\cvsubentry{Fall 2022}{Stochastic analysis, master.}
\cvsubentry{Fall 2021}{Stochastic analysis, master.}
% Rennes
\cventry{2016 --- 2019}{Université de Rennes 1}{France}{}{}{}
\cvsubentry{Spring 2019}{Computer practicals for engineering undergraduates, two sections.\newline%
                         Computer practicals for mathematics undergraduates.}
\cvsubentry{Fall 2018}{Computer practicals for engineering undergraduates, two sections.}
\cvsubentry{Fall 2017}{Computer practicals for engineering undergraduates, three sections.\newline%
                       Vector calculus for physics undergraduates}
\cvsubentry{Spring 2017}{Computer practicals for engineering undergraduates.\newline%
                         Linear algebra for economics undergraduates.}
\cvsubentry{Fall 2016}{Computer practicals for engineering undergraduates.}

\subsection{Organization and supervision}

% Luxembourg
\cventry{2021 -- 2022}{Université du Luxembourg}{}{}{}{}

\cvsubentry{Spring 2022}{Organization of a seminar course\newline%
                         {\small Oral presentations of students about selected topics, preparation supervised\newline%
                         by department members during the semester}\newline%
                         Bachelor project supervision, title:
                         \emph{Fine properties of Brownian motion.}\newline%
                         Oral presentation supervision, title: \emph{Equidecomposition of polytopes.}\newline%
                         Oral presentation supervision, title: \emph{Zeta regularization.}}
\cvsubentry{Spring 2023}{Computational mathematics project supervision, title:\newline%
                         \emph{\phantom{hello}Sharp transitions for percolation in Erd\H os--Renyi graphs.}\newline%
                         Summer project supervision, title: \emph{Random graphs.}}
\cvsubentry{Spring 2022}{Bachelor project supervision, title:\newline%
                         \emph{\phantom{hello}Poisson point processes and continuum percolation.}\newline%
                         Oral presentation supervision, title: \emph{Critical percolation in $\mathbb Z^2$.}\newline%
                         Computational mathematics project supervision, title:\newline%
                         \emph{\phantom{hello}Random tilings and the arctic circle theorem.}}

\section{Other scientific activities}

\cventry{2022 --- 2024}{Organizer of the internal probability seminar}{University of Luxembourg}{}{}{}
\cventry{2022}{Reviewing work for \emph{Journal of Nonlinear Sciences}}{}{}{}{}
\cventry{Dec 2018}{`5 minutes Lebesgue' talk}{Centre Henri Lebesgue}{Rennes}{}
   {A glimpse of rough paths theory, part of a series of (very) short talks.\newline Available on \href{https://youtu.be/r31qisjgbK8}{YouTube} (in French).}
\cventry{2018 --- 2019}
   {Organiser of the PhD seminar in probability}
   {IRMAR}{Rennes}{}
   {}
\cventry{Fall 2017}
   {Co-organiser of the PhD seminar in probability}
   {IRMAR}{Rennes}{}
   {In collaboration with Florian Lemonnier.}

\end{document}