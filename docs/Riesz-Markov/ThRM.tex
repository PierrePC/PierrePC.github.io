\documentclass{article}

\usepackage{pp}

\title{Théorème de Riesz-Markov}
\author{Pierre Perruchaud}
\date{Décembre 2016}

\begin{document}
\maketitle

Étant donné un espace topologique $S$, on note $\mathcal C_c(S)$ l'algèbre des fonctions réelles continues à support compact sur $S$, muni de la norme uniforme. On note $\mathcal C_c^+(S)$ le cône convexe des fonctions $f\in\mathcal C_c(S)$ partout positives.

On trouvera dans ce document une preuve du résultat suivant.

\begin{Thm}[de représentation de Riesz-Markov]~
\label{ThRM}

Soit $S$ un espace topologique séparé localement compact.

Soit $\Phi:\mathcal C_c^+(S)\to\R_+$ une application $R_+$-linéaire, c'est-à-dire telle que $\Phi(\lambda f+\mu g)=\lambda\Phi f+\mu\Phi g$ pour tous $\lambda,\mu\in\R_+$ et $f,g\in\mathcal C_c^+(S)$. 

Alors il existe une unique mesure borélienne quasi-régulière localement finie $\mu$ telle que pour tout $f\in\mathcal C_c^+(S)$,
$$\Phi f=\int_Sf\dif\mu\text.$$
\end{Thm}

On dit que la mesure $\mu$ représente la forme $\Phi$.

\begin{Cor}~
\label{CoPositif}

Soit $S$ un espace topologique séparé localement compact.

Soit $\Phi:\mathcal C_c(S)\to\R$ une application linéaire positive, c'est-à-dire telle que $\Phi f\geq0$ pour tout $f\in\mathcal C_c^+(S)$.

Alors il existe une unique mesure borélienne quasi-régulière localement finie $\mu$ telle que pour toute fonction $f\in\mathcal C_c(S)$, $\Phi f=\int_Sf\dif\mu$.
\end{Cor}

\begin{Cor}~
\label{CoBorne}

Soit $S$ un espace topologique séparé localement compact.

Soit $\Phi:\mathcal C_c(S)\to\R$ une application linéaire continue.

Alors il existe une unique mesure signée borélienne quasi-régulière localement finie $\mu$ telle que pour toute fonction $f\in\mathcal C_c(S)$, $\Phi f=\int_Sf\dif\mu$.
\end{Cor}

La démonstration du théorème \ref{ThRM} s'appuie sur un certain nombre de résultats de théorie de la mesure et de topologie cités en annexe. Elle se veut complète et efficace, peut-être au détriment d'une certaine pédagogie.

\section{Démonstration}

Introduisons tout d'abord quelques notations. Dans toute la suite, $\mathcal U$ désignera un ouvert, $K$ un compact, $F$ un fermé.

Pour tout $f\in\mathcal C_c(S)$ :
\begin{itemize}
\item $0\leq f$ (resp. $f\leq1$) signifie $f(x)\geq 0$ (resp. $f(x)\leq1$) pour tout $x\in S$ ;
\item pour tout $E\subset S$, $E\leq f$ (resp. $f\leq E$) signifie $\Ind{E}(x)\leq f(x)$ (resp. $f(x)\leq\Ind{E}(x)$) pour tout $x\in S$ ;
\item pour tout $E\subset S$, $f<E$ signifie $f\leq E$ et $\mathrm{Supp}(f)\subset E$.
\end{itemize}

\subsection{Construction d'une mesure}

On appelle mesure extérieure sur un ensemble $S$ toute application $\mu^*:\mathcal P(S)\to[0;+\infty]$ vérifiant
\begin{itemize}
\item $\mu^*(\varnothing)=0$ ;
\item si $A\subset B\subset S$, alors $\mu^*(A)\leq\mu^*(B)$ ;
\item pour toute famille $(A_n)_{n\in\N}$ de sous-ensembles de $S$, $\mu^*\left(\bigcup_{n\in\N}A_n\right)\leq\sum_{n\in\N}\mu^*(A_n)$. 
\end{itemize}

L'intérêt d'un tel objet vient du théorème \ref{MesureExt}, qui énonce que la restriction d'une mesure extérieure à une tribu bien choisie est une mesure. Construisons donc une mesure extérieure sur $S$.
\bigskip

On pose, pour tout ouvert $\mathcal U\subset S$, puis pour tout sous-ensemble $E\subset S$,
$$\mu_U^*(\mathcal U):=\sup_{0\leq f<\mathcal U}\Phi f$$
$$\mu^*(E):=\inf_{\mathcal U\supset E}\mu_U^*(\mathcal U)$$

\begin{Thm} L'application $\mu^*$ ainsi définie est une mesure extérieure. \end{Thm}

\begin{Dem}~

Notons que si $\mathcal U\subset\mathcal V$ sont deux ouverts de $S$, alors $\mu_U^*(\mathcal U)\leq\mu_U^*(\mathcal V)$ ; on en déduit que $\mu^*(\mathcal U)=\mu_U^*(\mathcal U)$ pour tout ouvert $\mathcal U\subset S$. On a évidemment $\mu^*(\varnothing)=\mu^*_U(\varnothing)=0$ et $\mu^*(A)\leq\mu^*(B)$ pour tous $A\subset B\subset S$. On doit donc, pour montrer que $\mu^*$ est une mesure extérieure, prouver la propriété de sous-additivité dénombrable.

\emph{Étape 1.} Soit $(\mathcal U_n)_{n\in\N}$ une famille dénombrable d'ouverts de $S$. Soit $f\in\mathcal C_c(S)$ une fonction telle que $0\leq f<\bigcup_{n\in\N}\mathcal U_n$.

Puisque $\mathrm{Supp}(f)$ est compact, et d'après le théorème \ref{Partition}, il existe des fonctions $\rho_0,\cdots,\rho_k\in\mathcal C_c(S)$ telles que $0\leq\rho_i<\mathcal U_i$ et $\sum_{0\leq i\leq k}\rho_i(x)=1$ pour tout $x\in\mathrm{Supp}(f)$. Ainsi, on peut donc majorer l'image de $f$ :
$$\Phi f=\Phi(f\rho_0)+\cdots+\Phi(f\rho_k)\leq\mu^*(\mathcal U_0)+\cdots+\mu^*(\mathcal U_k)\leq\sum_{n\in\N}\mu^*(\mathcal U_n)\text.$$

Ceci valant pour tout $f$, on a montré que $\mu^*(\bigcup_{n\in\N}\mathcal U_n)\leq\sum_{n\in\N}\mu^*(\mathcal U_n)$.

\emph{Étape 2.} Soit $(A_n)_{n\in\N}$ une famille dénombrable de sous-ensembles de $S$. Soit $\eps>0$.

Pour tout $n\in\N$, il existe un ouvert $\mathcal U_n\supset A_n$ tel que $\mu^*(\mathcal U_n)\leq\mu^*(A_n)+2^{-n}\eps$. On sait alors que
$$\mu^*\left(\bigcup_{n\in\N}A_n\right)
   \leq \mu^*\left(\bigcup_{n\in\N}\mathcal U_n\right)
   \leq \sum_{n\in\N}\mu^*(\mathcal U_n)
   \leq \sum_{n\in\N}\mu^*(A_n)+\sum_{n\in\N}2^{-n}\eps
   \leq 2\eps+\sum_{n\in\N}\mu^*(A_n)\text.$$

Ceci valant pour tout $\eps$, on a montré $\mu^*(\bigcup_nA_n)\leq\sum_n\mu^*(A_n)$, c'est-à-dire la sous-additivité dénombrable de $\mu^*$, ce qui conclut la démonstration.
\end{Dem}

Définissons une sous-partie $\mathcal F$ de $\mathcal P(S)$ par la propriété suivante :
$$\mathcal F=\left\{A\subset S\,|\,\forall B\subset S,
                    \mu^*(B)=\mu^*(A\cap B)+\mu^*(^cA\cap B)\right\}\text.$$
Alors, d'après le théorème \ref{MesureExt}, $\mathcal F$ est une tribu et $\mu:\mathcal F\to\R_+,A\mapsto\mu^*(A)$ est une mesure.

\begin{Thm} La tribu $\mathcal F$ contient les ouverts. \end{Thm}

La mesure $\mu$ est donc, à restriction près, borélienne, et les fonctions continues sont mesurables par rapport à cette tribu.

\begin{Dem}~

Notons que par sous-additivité, on sait déjà que $\mu^*(B)\leq\mu^*(A\cap B)+\mu^*(^cA\cap B)$ pour tous ensembles $A$ et $B$. Il suffit donc de montrer que $\mu^*(\mathcal U\cap A)+\mu^*(^c\mathcal U\cap A)\leq\mu^*(A)$ où $\mathcal U$ est ouvert et $A$ quelconque. Prouvons tout d'abord que pour tous ouverts $\mathcal U$ et $\mathcal V$,
$$\mu^*(\mathcal U\cap\mathcal V)+\mu^*(^c\mathcal U\cap\mathcal V)\leq\mu^*(\mathcal V)\text.$$

Soit $f\in\mathcal C_c(S)$ telle que $0\leq f<\mathcal U\cap\mathcal V$. Soit $g\in\mathcal C_c(S)$ telle que $0\leq g<\,^c\mathrm{Supp}(f)\cap\mathcal V$.

Alors $f+g$ vérifie $0\leq f+g<\mathcal V$ et
$$\Phi f+\Phi g=\Phi(f+g)\leq\mu^*(\mathcal V)$$
d'où, en considérant la borne supérieure sur $g$,
$$\Phi f+\mu^*(^c\mathcal U\cap\mathcal V) \leq \Phi f+\mu^*(^c\mathrm{Supp}(f)\cap\mathcal V)
                                           \leq \mu^*(\mathcal V)$$
et enfin, en prenant le supremum selon $f$,
$$\mu^*(\mathcal U\cap\mathcal V)+\mu^*(^c\mathcal U\cap\mathcal V)\leq\mu^*(\mathcal V)\text.$$

On conclut alors en considérant un ouvert $\mathcal U$ et un ensemble $A$. On constate que
$$\mu^*(\mathcal U\cap A)+\mu^*(^c\mathcal U\cap A)
    \leq \inf_{\mathcal V\supset A}\left( \mu^*(\mathcal U\cap \mathcal V)
                                        + \mu^*(^c\mathcal U\cap\mathcal V)
                                          \right)
    \leq \inf_{\mathcal V\supset A} \mu^*(\mathcal V)
       = \mu^*(A)\text,$$
et le résultat est démontré.
\end{Dem}

\subsection{Compatibilité}

On cherche à montrer dans cette partie que la mesure ainsi construite représente bien $\Phi$. Montrons tout d'abord un lemme.

\begin{Lem}~
\label{Inegalites}

Soit $f\in\mathcal C_c(S)$ vérifiant $0\leq f\leq1$. Alors
$$\mu(\left\{f=1\right\})\leq\Phi f\leq\mu(\mathrm{Supp}(f))\text,$$
$$\mu(\left\{f=1\right\})\leq\int_Sf\dif\mu\leq\mu(\mathrm{Supp}(f))\text.$$
\end{Lem}

En particulier, ceci prouve que $\mu$ est localement finie. En effet, si $K$ est compact, il existe d'après le théorème \ref{Normal} une fonction $K\leq f<S$ et $\mu(K)\leq\mu(\{f=1\})\leq\Phi f<\infty$. En supposant le lemme démontré, on a donc construit une mesure $\mu$ borélienne localemement finie, qui induit une forme $\R_+$-linéaire de $\mathcal C_c^+(S)$ dans $\R_+$ ; on montrera dans la suite de cette partie que cette forme coïncide avec $\Phi$.

\begin{Dem}[Démontration --- Lemme \ref{Inegalites}]~

La seconde propriété est élémentaire :
$$\mu(\left\{f=1\right\})=\int_S\Ind{f=1}\dif\mu\leq \int_Sf\dif\mu\leq\int_S\Ind{\mathrm{Supp}(f)}\dif\mu=\mu(\mathrm{Supp}(f))\text.$$

Montrons donc la première : soit $\eps>0$.
\begin{eqnarray*}
\mu(\left\{f=1\right\})
  &\leq& \mu(\left\{(1+\eps)f>1\right\})\\
  &   =& \sup_{0\leq g<\left\{(1+\eps)f>1\right\}}\Phi g\\
  &\leq& \sup_{0\leq g<\left\{(1+\eps)f>1\right\}}(1+\eps)\Phi f\\
  &   =& (1+\eps)\Phi f
\end{eqnarray*}
et la minoration est prouvée lorsque $\eps$ tend vers $0$.

Soit maintenant $\mathcal U\supset\mathrm{Supp}(f)$ un ouvert. Ainsi, $0\leq f<\mathcal U$ et $\Phi f\leq\mu(\mathcal U)$ ; la majoration est prouvée en passant à l'infimum sur les ouverts $\mathcal U$.
\end{Dem}

Le lemme nous permet alors de conclure.

Soit $f\in\mathcal C_c^+(S)$. Soit $\eps>0$. On définit pour tout $n\in\N$ la fonction $f_n:=0\vee\frac1\eps(f-n\eps)\wedge1\in\mathcal C_c^+(S)$ de sorte que $f=\eps\sum_{n\in\N}f_n$, où $f_n=0$ à partir d'un certain rang $N_\eps\in\N$. Alors
\begin{eqnarray*}
\Phi f-\int_Sf\dif\mu
  &   =& \eps\sum_{0\leq n\leq N_\eps}\left(\Phi f_n-\int_Sf_n\dif\mu\right)\\
  &\leq& \eps\sum_{0\leq n\leq N_\eps}\left(\mu(\mathrm{Supp}(f_n))-\mu(\left\{f_n=1\right\})\right)\\
  &   =& \eps\sum_{0\leq n\leq N_\eps}\left( \mu\left(\overline{\left\{f>n\eps\right\}}\right)
                                           - \mu\left(\left\{f\geq (n+1)\eps\right\}\right)
                                      \right)\\
  &   =& \eps\cdot\mu\left(\bigcup_{0\leq n\leq N_\eps} \overline{\left\{f>n\eps\right\}}
                                         \setminus \left\{f\geq (n+1)\eps\right\}\right)\\
  &\leq& \eps\mu\left(\mathrm{Supp}(f)\right)
\end{eqnarray*}
d'où $\Phi f\leq\int_Sf\dif\mu$ puisque $\eps>0$ est quelconque et $\mathrm{Supp}(f)$ est compact donc de mesure finie. Réciproquement,
\begin{eqnarray*}
\int_Sf\dif\mu-\Phi f
  &   =& \eps\sum_{0\leq n\leq N_\eps}\left(\int_Sf_n\dif\mu-\Phi f_n\right)\\
  &\leq& \eps\sum_{0\leq n\leq N_\eps}\left(\mu(\mathrm{Supp}(f_n))-\mu(\left\{f_n=1\right\})\right)\\
  &\leq& \eps\mu\left(\mathrm{Supp}(f)\right)
\end{eqnarray*}
d'où $\Phi f\geq \int_Sf\dif\mu$. On a donc conclu : pour toute fonction $f\in\mathcal C_c^+(S)$,
$$\Phi f=\int_Sf\dif\mu\text.$$

\subsection{Propriétés de la mesure}

Il reste, pour achever la démonstration du théorème \ref{ThRM}, à montrer que $\mu$ est quasi-régulière, et qu'elle est unique.

\begin{Thm} La mesure $\mu$ est quasi-régulière. \end{Thm}

\begin{Dem}~

On a montré que $\mu$ est borélienne, et par construction $\mu(A)=\inf_{\mathcal U\supset A}\mu(\mathcal U)$. Il reste donc à montrer que $\mu(\mathcal U)=\sup_{K\subset \mathcal U}\mu(K)$ où $\mathcal U$ est un ouvert quelconque et $K$ parcourt les compacts de $\mathcal U$.
Tout d'abord, si $0\leq f<\mathcal U$,
$$\Phi f=\int_Sf\dif\mu\leq\mu(\mathrm{Supp}(f))\leq \sup_{K\subset\mathcal U}\mu(K)$$
et on en déduit $\mu(\mathcal U)\leq\sup_{K\subset \mathcal U}\mu(K)$ en passant à la borne supérieure sur $f$. Réciproquement, si $K\subset\mathcal U$ est compact, il existe d'après le théorème \ref{Normal} une fonction $K\leq f<\mathcal U$, et
$$\mu(K)\leq\mu(\{f=1\})\leq\Phi f\leq\mu(\mathcal U)$$
d'où $\sup_{K\subset \mathcal U}\mu(K)\leq\mu(\mathcal U)$, ce qui prouve $\mu(\mathcal U)=\sup_{K\supset\mathcal U}\mu(K)$ et conclut la démonstration.
\end{Dem}

Il est clair qu'une mesure quasi-régulière $\nu$ sur $S$ est caractérisée par ses valeurs sur les compacts, puisque $\nu(A)=\inf_{\mathcal U\supset A}\sup_{K\subset\mathcal U}\nu(K)$. Il devient alors naturel de s'intéresser au résultat suivant.

\begin{Thm}
Si $\nu$ resprésente $\Phi$ et $\nu(\mathcal U)=\sup_{K\subset\mathcal U}\nu(K)$, alors
$$\nu(K)=\inf_{K\leq f}\Phi f\text.$$
\end{Thm}

En particulier, si deux mesures quasi-régulières représentent $\Phi$, elles coïncident sur les compacts, donc sont égales. Ceci montre l'unicité de la mesure, et conclut la démonstration du théorème \ref{ThRM}.

\begin{Dem}~

Soit $\nu$ une telle mesure. Un sens de l'inégalité est trivial :
$$\nu(K)\leq\int_Sf\dif\nu=\Phi f$$
pour toute fonction $f\in\mathcal C_c(S)$ telle que $K\leq f$, et $\nu(K)\leq\inf_{K\leq f}\Phi f$.

Réciproquement, soit $\eps>0$. D'après le théorème \ref{OuvertCompact}, il existe un ouvert $\mathcal U$ d'adhérence compacte tel que $K\subset\mathcal U$. Ainsi, $\mathcal U\setminus K$ est un ouvert de mesure finie, donc il existe un compact $K'\subset\mathcal U\setminus K$ tel que
$$\nu(K')+\eps\geq\nu(\,\mathcal U\setminus K)\text.$$
D'après le théorème \ref{Normal}, il existe $K\leq f<\mathcal U\setminus K'$ et
$$\Phi f \leq \nu(\mathcal U\setminus K')
            = \nu(\mathcal U)-\nu(K')
            = \nu(\mathcal U\setminus K)+\nu(K)-\nu(K')
         \leq \nu(K)+\eps\text.$$
En passant à l'infimum sur $f$, on obtient $\inf_{K\leq f\leq1}\Phi f\leq\nu(K)+\eps$. Ceci valant pour tout $\eps>0$, on a montré que $\inf_{K\leq f\leq1}\Phi f\leq\nu(K)$, et l'égalité est montrée.
\end{Dem}

\section{Corollaires}
\subsection{Représentation des formes linéaires positives}

Rappelons le corollaire \ref{CoPositif}. Soit $S$ un espace topologique séparé localement compact, et $\Phi:\mathcal C_c(S)\to\R_+$ une forme linéaire positive. Alors $\Phi$ est représentée de manière unique par une mesure borélienne quasi-régulière localement finie.

Soient donc $S$ et $\Phi$ vérifiant ces hypothèses. Alors la restriction de $\Phi$ à $\mathcal C_c^+(S)$ est $\R_+$-linéaire, et d'après le théorème \ref{ThRM}, il existe une mesure borélienne quasi-régulière localement finie telle que $\Phi f=\int_Sf\dif\mu$ pour toute fonction $f\in\mathcal C_c^+(S)$. Ainsi, pour toute fonction $f\in\mathcal C_c(\R)$, les fonctions $|f|$ et $|f|-f$ sont dans $\mathcal C_c^+(S)$, et
$$\Phi f = \Phi|f|-\Phi\left(|f|-f\right)
         = \int_S|f|\dif\mu-\int_S\left(|f|-f\right)\dif\mu
         = \int_Sf\dif\mu\text.$$

On a ainsi existence. L'unicité est facilement vérifiée ; en effet, toute autre mesure $\nu$ borélienne quasi-régulière localement finie resprésentant $\Phi$ doit aussi représenter sa restriction à $\mathcal C_c^+(S)$, et $\mu=\nu$ d'après le théorème \ref{ThRM}.

\subsection{Représentation des formes linéaires positives}

Le corollaire \ref{CoBorne} est plus délicat ; rappelons-en l'énoncé. Soit $S$ un espace topologique séparé localement compact, et $\Phi:\mathcal C_c(S)\to\R$ continue pour la norme infinie. Alors $\Phi$ est représentée de manière unique par une mesure signée borélienne quasi-régulière localement finie.

Soient donc $S$ et $\Phi$ vérifiant ces hypothèses. On définit $|\Phi|:\mathcal C_c^+(S)\to\R_+$ par
$$|\Phi|f=\sup_{|g|\leq f}\Phi g$$
où le supremum est pris sur les fonctions $g\in\mathcal C_c(S)$ telles que $|g(x)|\leq f(x)$ pour tout $x\in S$. Notons que $|0|\leq f$, donc $|\Phi|$ est bien à valeurs positives, et que $|g|_\infty\leq|f|_\infty<\infty$ lorsque $|g|\leq f$, donc le supremum est toujours fini puisque $\Phi$ est continue.

\begin{Thm} Les applications $|\Phi|$ et $|\Phi|-\Phi:\mathcal C_s^+(S)\to R_+$ sont bien définies et $\R_+$-linéaires. \end{Thm}

\begin{Dem}~

On montre tout d'abord que $|\Phi|$ est $\R_+$-linéaire. Si $\lambda\in\R_+$ et $f\in\mathcal C_c^+(S)$, alors
$$|\Phi|(\lambda f) = \sup_{|g|\leq\lambda f}\Phi g
                    = \sup_{|g|\leq f}\Phi(\lambda g)
                    = \lambda\sup_{|g|\leq f}\Phi f
                    = \lambda|\Phi|f\text.$$
Soient maintenant $f_1$ et $f_2$ deux fonctions de $\mathcal C_c^+(S)$. On pose $f=f_1+f_2$. Si $g_1,g_2\in\mathcal C_c(S)$ vérifient $|g_1|\leq f_1$ et $|g_2|\leq f_2$, alors $|g_1+g_2|\leq f_1+f_2$ et
$$\Phi g_1+\Phi g_2\leq\Phi(g_1+g_2)\leq|\Phi|(f_1+f_2)\text,$$
d'où, en considérant la borne supérieure sur $g_1$ et $g_2$, $|\Phi|f_1+|\Phi|f_2\leq|\Phi|f$. Réciproquement, si $g\in\mathcal C_c(S)$ vérifie $|g|\leq f$, on pose
$g_1:=f_1g/f$ et $g_2:=f_2g/f$,
avec la convention $g_1=0$ là où $f$ (donc $g$) s'annule. On vérifie que les fonctions $g_1$ et $g_2$ appartiennent alors à l'ensemble $\mathcal C_c(S)$. De plus, elles vérifient $|g_1|\leq f_1$ et $|g_2|\leq f_2$ avec $g=g_1+g_2$. On peut écrire
$$\Phi g=\Phi g_1+\Phi g_2\leq|\Phi|f_1+|\Phi|f_2$$
et en prenant le supremum sur $g$, $|\Phi|f\leq|\Phi|f_1+|\Phi|f_2$. On a donc vérifié que $|\Phi|$ est $\R_+$-linéaire.

L'application $|\Phi|-\Phi$ est bien définie, puisque pour toute fonction $f\in\mathcal C_c^+(S)$, $|f|\leq f$ d'où $$\left(|\Phi|-\Phi\right)f\geq \Phi f-\Phi f=0\text.$$ Elle est $\R_+$-linéaire puisque $\Phi$ est linéaire et $|\Phi|$ est $\R_+$-linéaire.
\end{Dem}

Il existe donc deux mesures (positives) boréliennes quasi-régulières localement finies $|\mu|$ et $\nu$ telles que pour toute fonction $f\in\mathcal C_c^+(S)$,
$$\int_Sf\dif|\mu|=|\Phi|f\text{ et }\int_Sf\dif\nu=\left(|\Phi|-\Phi\right)f\text.$$

\begin{Thm}~

En notant $\|\cdot\|$ la norme d'opérateur, $\frac12\nu(S)\leq|\mu|(S)=\|\Phi\|$.
En particulier, ces mesures sont finies.
\end{Thm}

Concluons l'existence en admettant pour l'instant ce théorème. On peut poser $\mu$ la mesure signée $|\mu|-\nu$, qui est donc borélienne quasi-régulière et localement finie. Ainsi, pour tout $f\in\mathcal C_c^+(S)$,
$$\Phi f=|\Phi|f-\left(|\Phi|-\Phi\right)f=\int_Sf\dif|\mu|-\int_Sf\dif\nu=\int_Sf\dif\mu\text,$$
et pour tout $f\in\mathcal C_c(S)$,
$$\Phi f=\Phi|f|-\Phi(|f|-f)=\int_S|f|\dif\mu-\int_S\left(|f|-f\right)\dif\mu=\int_Sf\dif\mu\text.$$

On montre maintenant le théorème.

\begin{Dem}~

Notons tout d'abord que si $\eta$ est une mesure (positive) quasi-régulière sur $S$, alors la masse totale de la mesure est donnée par $\eta(S)=\sup_{0\leq f\leq1}\int_Sf\dif\eta$. En effet, on sait que $\eta(S)=\sup_{K}\eta(K)$, où $K$ parcourt les compacts de $S$. Or, pour tout compact $K$, il existe d'après le théorème \ref{Normal} une fonction $K\leq f\leq1$ dans $\mathcal C_c(S)$ ; réciproquement, pour toute fonction $f\in\mathcal C_c(S)$ vérifiant $0\leq f\leq1$, $\int_Sf\dif\eta\leq\eta(\mathrm{Supp}(f))$. Ainsi,
$$\sup_K\eta(K)\leq\sup_{0\leq f\leq1}\int_Sf\dif\eta\leq\sup_K\eta(K)\text,$$
et $\eta(S)=\sup_{0\leq f\leq1}\int_Sf\dif\eta$ comme annoncé.
\bigskip

On en déduit la masse de $|\mu|$.
$$|\mu|(S) = \sup_{0\leq f\leq1}\int_Sf\dif|\mu|
           = \sup_{0\leq f\leq1}|\Phi|f
           = \sup_{0\leq f\leq1}\sup_{|g|\leq f}\Phi g
           = \sup_{|g|\leq 1}\Phi g
           = \|\Phi\|$$

Majorons la masse de $\nu$ ; soit $f\in\mathcal C_c(S)$ telle que $0\leq f\leq1$.
$$\int_Sf\dif\nu    = \left(|\Phi|-\Phi\right)f
                    = \sup_{|g|\leq f}\Phi g-\Phi f
                 \leq \sup_{|g|\leq1}\Phi g+\|\Phi\|
                 \leq \|\Phi\|+\|\Phi\|
                    = 2\|\Phi\|\text,$$
ce qui conclut la démonstration du théorème.
\end{Dem}

Il reste enfin à montrer l'unicité de cette mesure. La démonstration est essentiellement indépendante de l'existence ; mettons l'emphase sur ce point en reformulant le résultat d'unicité.

Soient $S$ un espace topologique séparé localement compact, et $\Phi:\mathcal C_c(S)\to\R$ une forme linéaire continue. Soient $\mu$ et $\nu$ deux mesures signées boréliennes quasi-régulières localement finies qui représentent $\Phi$. Il nous faut montrer que $\mu=\nu$.

La mesure signée $\mu$ est borélienne quasi-régulière localement finie ; il existe donc deux mesures (positives) $\mu^+$ et $\mu^-$ boréliennes quasi-régulières localement finies telles que $\mu=\mu^+-\mu^-$. On construit de même $\nu^+$ et $\nu^-$. Définissons les formes $F,G:\mathcal C_c^+(S)\to\R_+$ par
$$Ff:=\int_Sf\dif\mu^++\int_Sf\dif\nu^-\text{ et }Gf:=\int_Sf\dif\nu^++\int_Sf\dif\mu^-\text.$$
Ce sont des formes $R_+$-linéaires, représentées respectivement par les mesures (positives) boréliennes quasi-régulières localement compactes $\mu^++\nu^-$ et $\nu^++\mu^-$.

Or, pour toute fonction $f\in\mathcal C_c^+(S)$,
\begin{align*}
Ff &= \int_Sf\dif\mu^++\int_Sf\dif\nu^-
    = \left(\int_Sf\dif\mu\,+\int_Sf\dif\mu^-\right)+\int_Sf\dif\nu^-\\
   &= \Phi f+\int_Sf\dif\nu^-+\int_Sf\dif\mu^-\\
   &= \left(\int_Sf\dif\nu^+-\int_Sf\dif\nu^-\right)+\int_Sf\dif\nu^-+\int_Sf\dif\mu^-
    = \int_Sf\dif\nu^++\int_Sf\dif\mu^-\\
   &= Gf\text.
\end{align*}
Ces deux formes $\R_+$-linéaires sont égales ; la mesure qui les représente est donc unique sous les hypothèses du théorèmes \ref{ThRM} qui sont vérifiées ici, et $\mu^++\nu^-=\nu^++\mu^-$. Il suffit alors de constater que
$$\mu=\mu^+-\mu^-=\nu^+-\nu^-=\nu$$
pour conclure quand à l'unicité de la mesure et achever la démonstration du corollaire \ref{CoBorne}.

\section{Prérequis}
\subsection{Topologie}

\begin{Thm}~
\label{OuvertCompact}

Soit $S$ un espace topologique séparé localement compact, $K$ un compact de $S$.

Il existe un compact $K'$ tel que $K\subset\mathring{K'}$. De manière équivalente, il existe un ouvert $\mathcal U\supset K$ d'adhérence compacte.
\end{Thm}

\begin{Thm}[Lemme d'Urysohn]~
\label{Normal}

Soit $S$ un espace topologique séparé localement compact. Soient $K$ et $\mathcal U$ un compact et un ouvert tels que $K\subset\mathcal U$.

Alors il existe $f\in\mathcal C_c(S)$ telle que $K\leq f<\mathcal U$.
\end{Thm}

\begin{Thm}[Partitions de l'unité]~
\label{Partition}

Soit $S$ un espace topologique séparé localement compact. Soit $K\subset S$ un compact recouvert par une famille d'ouverts $(\mathcal U_i)_{i\in I}$.

Alors il existe des fonctions $\rho_1,\cdots,\rho_n\in\mathcal C_c(S)$ et des indices $i_1,\cdots,i_n\in I$ tels que
\begin{itemize}
\item pour tout $1\leq k\leq n$, $0\leq\rho_k<\mathcal U_{i_k}$ ;
\item pour tout $x\in K$, $\sum_{1\leq k\leq n}\rho_k(x)=1$.
\end{itemize}
\end{Thm}

\subsection{Théorie de la mesure}

\begin{Thm}[Carathéodory]~
\label{MesureExt}

Soit $S$ un ensemble muni d'une mesure extérieure $\mu^*$. Soit $\mathcal F\subset\mathcal P(S)$ définie par
$$\mathcal F=\left\{E\subset S\,|\,\forall A\subset S,\mu^*(A)=\mu^*(A\cap E)+\mu^*(A\cap\,^c\!E)\right\}\text.$$

Alors $\mathcal F$ est une tribu et $\mu:\mathcal F\to R_+,E\mapsto\mu^*(E)$ est une mesure sur $(S,\mathcal F)$.
\end{Thm}

\section{Références}

Ce document a été écrit sans référence. Le lecteur intéressé pourra cependant trouver une preuve de ce résultat dans l'ouvrage de D. \textsc{Cohn} ou celui de P. \textsc{Halmos}, s'intitulant tous deux \textit{Measure Theory}.

Les prérequis utilisés sont des faits bien connus, dont les énoncés et références associées pourront par exemple se trouver sur Wikipédia, au voisinage des articles \emph{Espace localement compact}, \emph{Lemme d'Urysohn}, \emph{Partition de l'unité}, \emph{Mesure extérieure} ou de leurs homologues anglophones.

%positive implique croissante
%\mu^*_U
%borélienne à restriction près.

\end{document}
